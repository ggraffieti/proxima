% !TeX spellcheck = it_IT
\documentclass[a4paper,12pt]{report}

\usepackage{alltt, fancyvrb, url}
\usepackage{graphicx}
\usepackage{algorithmic}
\usepackage[utf8]{inputenc}
\usepackage{titling}
\usepackage{fancyhdr}
\usepackage{fontenc}
\usepackage{amsmath,mathtools,algorithm}
\usepackage{amssymb}
\usepackage{longtable}
\usepackage{setspace}
\usepackage{listings}
\usepackage{color}


\usepackage[hidelinks]{hyperref}

\usepackage[italian]{babel}
\usepackage[italian]{cleveref}

\lstdefinestyle{json}
{
	basicstyle = \footnotesize\ttfamily,
	frame = single,
	tabsize = 4
}

\lstdefinelanguage{myScala}{
	morekeywords={abstract,case,catch,class,def,%
		do,else,extends,false,final,finally,%
		for,if,implicit,import,match,mixin,%
		new,null,object,override,package,%
		private,protected,requires,return,sealed,%
		super,this,throw,trait,true,try,%
		type,val,var,while,with,yield},
	basicstyle=\ttfamily\small,
	sensitive=true,
	keywordstyle=\color{blue},
	morecomment=[l]{//},
	morecomment=[n]{/*}{*/},
	morestring=[b]",
	morestring=[b]',
	morestring=[b]"""
}


% MARGINI LARGHI
%\textwidth 6.3 in % Width of text line.
%    \textheight 9.2 in
%    \oddsidemargin 0 in      %   Left margin on odd-numbered pages.
%    \evensidemargin 0 in      %   Left margin on even-numbered pages.
%    \topmargin 0.2 in
%    \headheight 0 in       %   Width of marginal notes.
%    \headsep 0 in
%    \topskip 0 in

\pretitle{%
	\begin{center}
		\LARGE
	}
\posttitle{\end{center}}


\title{\Huge \textbf{Proxima} \\
	\vspace{10pt}
	\vspace{20pt}
}
\author{
	Gabriele Graffieti \\ \small \url{gabriele.graffieti@studio.unibo.it}
	\vspace{15pt}
	\\
	Alfredo Maffi \\ \small \url{alfredo.maffi@studio.unibo.it}
	\vspace{15pt}
	\\
	Manuel Peruzzi \\ \small \url{manuel.peruzzi@studio.unibo.it}
}

\date{}

\pagestyle{fancy}
\fancyhf{}
\fancyhead[L]{\ifthenelse{\isodd{\value{page}}}{\thepage}{\leftmark}}
\fancyhead[R]{\ifthenelse{\isodd{\value{page}}}{\leftmark}{\thepage}}
\renewcommand{\chaptermark}[1]{\markboth{#1}{}} 

\begin{document}

\maketitle
\pagenumbering{arabic}


\tableofcontents

\chapter{Introduzione}

\chapter{Idea}
\section{Visione}

\section{Goal}
 
\chapter{Analisi dei requisiti} 
\section{Idea}
Il sistema \emph{Proxima} nasce dalla necessità di fornire un supporto, in caso di emergenza, ai soccorritori accorsi sul posto. Ogni persona dovrà essere munita di un dispositivo di identificazione, che potrà essere letto dal soccorritore, tramite l'utilizzo di un apposito lettore. Tale lettore dovrà essere provvisto di schermo, sul quale saranno visualizzati i dati medici del paziente, in modo da consentire interventi mirati e con maggiore sicurezza. 

\section{Requisiti funzionali}
\begin{description}
	\item \textbf{Identificatore}: ogni utilizzatore del servizio dovrà essere costantemente in possesso di tale dispositivo, che dovrà quindi essere compatto, non ingombrante e possibilmente indossabile. Esso dovrà avere le seguenti funzionalità: 
	\begin{itemize}
		\item Identificare l'utente in maniera univoca.
		\item La trasmissione dell'identificativo dovrà avvenire in modalità \emph{wireless}, con un raggio limitato (1-3 metri).
		\item Lo scambio dell'identificativo dovrà avvenire in maniera trasparente all'utente, senza consensi espliciti ad ogni trasmissione.
	\end{itemize}

	\item \textbf{Lettore}: dispositivo che dovrà essere in possesso del soccorritore, in modo che possa identificare e mostrare i dati degli utenti. Per fare ciò dovrà essere in grado di comunicare sia con gli identificatori, sia con la banca dati. Il dispositivo dovrà avere le seguenti funzionalità:
	\begin{itemize}
		\item Permettere la ricerca degli identificatori nelle vicinanze (1-3 metri). La ricerca dovrà essere esplicitamente avviata dal soccorritore, e dovrà essere possibile interromperla.
		\item Nel caso durante la fase di ricerca siano stati trovati degli identificatori, il lettore dovrà essere in grado di recuperare l'identificativo dei relativi possessori.
		\item Contattare la banca dati remota in modo da acquisire i dati medici relativi al paziente identificato dall'identificativo recuperato in precedenza.
		\item Attraverso il lettore dovrà essere possibile identificare in maniera univoca il soccorritore, in modo che il sistema possa decidere se il soccorritore sia autorizzato alla lettura delle informazioni mediche dell'utente.
		\item Mostrare le informazioni mediche del paziente al soccorritore.
	\end{itemize}
\end{description}

I dati medici degli utenti sono forniti sotto forma di banca dati elettronica, accessibile attraverso la rete internet. All'interno della banca dati fornita sono conservate anche le informazioni sui turni dei soccorritori. 

Il sistema dovrà definire una politica di accesso ai dati contenuti nella banca dati sulla base dell'identità dei soccorritori e dei turni di lavoro. 

\section{Glossario}
In questa sezione verranno esplorati ed esplicati i concetti chiave usati dal committente durante la fase di raccolta dei requisiti. 
\begin{itemize}
	\textbf{Soccorritore}: 
\end{itemize}

\section{Scenari}
\subsection{Scenario I}
In una situazione di emergenza, il soccorritore arriva sul posto, apre l'applicazione sul proprio lettore e lo avvicina al proprio identificatore, in modo da autenticarsi nel sistema. Subito dopo il soccorritore avvicina il lettore all'identificatore del paziente e riceverà i dati medici dello stesso. 
\subsection{Scenario II}
In una situazione di emergenza, il soccorritore arriva sul posto, apre l'applicazione sul proprio lettore e lo avvicina al proprio identificatore, in modo da autenticarsi nel sistema. Subito dopo il soccorritore avvicina il lettore all'identificatore del paziente per richiedere i dati medici dello stesso. Nel caso essi non siano presenti nel sistema, dovrà essere mostrato un messaggio di errore.
\subsection{Scenario III} 
In una situazione di emergenza, il soccorritore arriva sul posto e apre l'applicazione sul proprio lettore. Nel caso non sia possibile creare alcuna connessione con il server, deve essere mostrato un messaggio che avvisa il soccorritore di intervenire ``manualmente'', senza contare sul servizio.


\section{Requisiti non funzionali}
\begin{itemize}
	\item Sicurezza (dei dati e delle comunicazioni), resistente ad attacchi quali \emph{man-in-the-middle}, \emph{replay attack}, \emph{sniffing}\dots 
	\item \emph{Non-repudation} delle azioni fatte, identificazione univoca di tutti i soggetti (medico, utente). 
	\item Availability (del server e dell'identificatore).
	\item Scalability del server
	\item Fault tolerance (server, batteria scarica identificatore, mancanza di connessione internet).
\end{itemize}

\section{Domain model}

\section{Test plan}

\chapter{Analisi del problema}

\section{Studio di fattibilità}

\section{Abstraction gap}

\section{Problematiche}

\section{Analisi dei rischi}

\section{Architettura logica (?)}
Il sistema risulta essere composto dalle seguenti componenti:
\begin{itemize}
	\item Server
	\item Lettore
	\item Identificatore
\end{itemize}

\subsection{Scenario I}
In una situazione di emergenza, il soccorritore arriva sul posto, apre l'applicazione sul proprio lettore e lo avvicina al proprio identificatore, in modo da autenticarsi nel sistema. Subito dopo il soccorritore avvicina il lettore all'identificatore del paziente e riceverà i dati medici dello stesso. 
\subsection{Scenario II}
In una situazione di emergenza, il soccorritore arriva sul posto, apre l'applicazione sul proprio lettore e lo avvicina al proprio identificatore, in modo da autenticarsi nel sistema. Subito dopo il soccorritore avvicina il lettore all'identificatore del paziente per richiedere i dati medici dello stesso. Nel caso essi non siano presenti nel sistema, dovrà essere mostrato un messaggio di errore.
\subsection{Scenario III} 
In una situazione di emergenza, il soccorritore arriva sul posto e apre l'applicazione sul proprio lettore. Nel caso non sia possibile creare alcuna connessione con il server, deve essere mostrato un messaggio che avvisa il soccorritore di intervenire ``manualmente'', senza contare sul servizio.



\chapter{Piano di lavoro}

\chapter{Progettazione}

\chapter{Implementazione}

\chapter{Testing} 

\chapter{Deployment (?)}

\chapter{Manutenzione}

\chapter{Restrospettiva}
\section{Giudizio finale}
\section{Sviluppi futuri}

 
\end{document}
