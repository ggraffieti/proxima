% !TeX spellcheck = it_IT
\documentclass[a4paper,12pt]{report}

\usepackage{alltt, fancyvrb, url}
\usepackage{graphicx}
\usepackage{algorithmic}
\usepackage[utf8]{inputenc}
\usepackage{titling}
\usepackage{fancyhdr}
\usepackage{fontenc}
\usepackage{amsmath,mathtools,algorithm}
\usepackage{amssymb}
\usepackage{longtable}
\usepackage{setspace}
\usepackage{listings}
\usepackage{color}


\usepackage[hidelinks]{hyperref}

\usepackage[italian]{babel}
\usepackage[italian]{cleveref}

\lstdefinestyle{json}
{
	basicstyle = \footnotesize\ttfamily,
	frame = single,
	tabsize = 4
}

\lstdefinelanguage{myScala}{
	morekeywords={abstract,case,catch,class,def,%
		do,else,extends,false,final,finally,%
		for,if,implicit,import,match,mixin,%
		new,null,object,override,package,%
		private,protected,requires,return,sealed,%
		super,this,throw,trait,true,try,%
		type,val,var,while,with,yield},
	basicstyle=\ttfamily\small,
	sensitive=true,
	keywordstyle=\color{blue},
	morecomment=[l]{//},
	morecomment=[n]{/*}{*/},
	morestring=[b]",
	morestring=[b]',
	morestring=[b]"""
}


% MARGINI LARGHI
%\textwidth 6.3 in % Width of text line.
%    \textheight 9.2 in
%    \oddsidemargin 0 in      %   Left margin on odd-numbered pages.
%    \evensidemargin 0 in      %   Left margin on even-numbered pages.
%    \topmargin 0.2 in
%    \headheight 0 in       %   Width of marginal notes.
%    \headsep 0 in
%    \topskip 0 in

\pretitle{%
	\begin{center}
		\LARGE
	}
\posttitle{\end{center}}


\title{\Huge \textbf{Proxima} \\
	\vspace{10pt}
	\vspace{20pt}
}
\author{
	Gabriele Graffieti \\ \small \url{gabriele.graffieti@studio.unibo.it}
	\vspace{15pt}
	\\
	Alfredo Maffi \\ \small \url{alfredo.maffi@studio.unibo.it}
	\vspace{15pt}
	\\
	Manuel Peruzzi \\ \small \url{manuel.peruzzi@studio.unibo.it}
}

\date{}

\pagestyle{fancy}
\fancyhf{}
\fancyhead[L]{\ifthenelse{\isodd{\value{page}}}{\thepage}{\leftmark}}
\fancyhead[R]{\ifthenelse{\isodd{\value{page}}}{\leftmark}{\thepage}}
\renewcommand{\chaptermark}[1]{\markboth{#1}{}} 

\begin{document}

\maketitle
\pagenumbering{arabic}


\tableofcontents

\chapter{Introduzione}

\chapter{Idea}
\section{Visione}
\section{Goal}
 
\chapter{Analisi dei requisiti} 

\section{Requisiti funzionali}
\begin{description}
	\item \textbf{Dispositivo utente}: ogni utilizzatore del servizio dovrà essere costantemente in possesso di tale dispositivo, che dovrà quindi essere compatto, non ingombrante e possibilmente indossabile. Esso dovrà avere le seguenti funzionalità: 
	\begin{itemize}
		\item  Identificare l'utente in maniera univoca. Il contenuto informativo presente al suo interno dovrà essere il minimo necessario per una corretta identificazione. 
		\item La trasmissione dell'identificativo dovrà essere eseguita soltanto quando richiesto dal lettore. La trasmissione dovrà avvenire in modalità \emph{wireless}, con un raggio limitato (1-3 metri).
		\item Dovrà essere impossibile risalire all'identità della persona attraverso il solo identificativo inviato, per che soggetti non autorizzati possano tracciare i movimenti degli utenti. 
		\item Non potrà essere possibile, da parte di un utente, assumere l'identità di un'altra persona, intercettando il suo identificativo e spacciandolo come proprio (\emph{replay attack}).
		\item Lo scambio dell'identificativo dovrà avvenire in maniera trasparente all'utente, senza consensi espliciti ad ogni trasmissione.
		\item Nel caso il dispositivo venga perso/rubato/compromesso, dovrà essere possibile il blocco o la sostituzione dell'identificativo dell'utente, rivolgendosi alle autorità competenti. 
	\end{itemize}

	\item \textbf{Lettore}: dispositivo Android che dovrà essere in possesso del soccorritore, in modo che possa identificare e richiedere i dati degli utenti. Per fare ciò dovrà essere in grado di comunicare sia con i dispositivi in possesso degli utenti, sia con il server centrale. Il dispositivo dovrà avere le seguenti funzionalità:
	\begin{itemize}
		\item Permettere la ricerca dei dispositivi utente nelle vicinanze (1-3 metri). La ricerca dovrà essere esplicitamente avviata dal soccorritore, dopo l'inserimento di un pin/password per la conferma della sua identità. La ricerca potrà essere interrotta dal soccorritore o in modo automatico dopo $x$ minuti.
		\item Nel caso durante la fase di ricerca siano stati trovati dei dispositivi utente, essi dovranno essere contattati in modo da ricevere l'identificativo dei relativi possessori.
		\item L'identificativo così ottenuto dovrà essere inviato al server centrale, che, nel caso il soccorritore sia autorizzato alla lettura dei dati, risponderà con le informazioni mediche dell'utente. 
		\item Il lettore dovrà identificare in maniera univoca il soccorritore all'interno del sistema, in modo che il server possa decidere se esso sia autorizzato alla lettura delle informazioni mediche dell'utente. 
		\item Le informazioni mediche dovranno essere mostrate al soccorritore in modo che siano facilmente consultabili.
		\item Nessun dato relativo agli utenti dovrà essere memorizzato nel dispositivo di lettura in modo permanente. 
	\end{itemize}

	\item \textbf{Server}: esso è responsabile del mantenimento dei dati medici degli utenti. Esso dovrà processare le richieste provenienti dai lettori, e autorizzare o meno la lettura dei dati. Il server avrà le seguenti funzionalità:
	\begin{itemize}
		\item La politica di accesso ai dati dovrà essere di tipo \emph{default-deny}, ovvero nessuno può accedere ai dati degli utenti senza esplicita autorizzazione da parte dell'ospedale competente. 
		\item Dovrà essere possibile designare un dato soccorritore per la gestione di una data emergenza. In questo modo il soccorritore potrà avere accesso ai dati degli utenti per periodo di tempo limitato ()
		\item Solamente i soccorritori designati per la gestione della specifica emergenza potranno accedere ai dati dei soggetti coinvolti. Personale non autorizzato o altri soccorritori non avranno alcun modo di risalire neppure all'identità di tali soggetti.
		\item Esso dovrà stabilire l'identità del soccorritore in modo univoco, in modo da poterlo autorizzare o meno alla lettura dei dati.  
		\item Ogni accesso ai dati dovrà essere tracciato in modo sicuro, per garantire \emph{accountability} e \emph{non-repudiation} degli accessi.
		\item I dati all'interno del server dovranno essere reperiti dalle cartelle elettroniche, già presenti nei sistemi informativi dei servizi sanitari. 
	\end{itemize}
	
	
	
	
	
	Solamente chi è autorizzato può identificare l'utente. esso deve decidere se un certo soccorritore può accedere ai dati di un soggetto ed a quali dati può accedere. Per fare ciò esso deve stabilire l'identità del soccorritore in modo univoco. Ogni accesso ai dati dovrà essere loggato in modo sicuro e in modo da da garantire \emph{accountability} e \emph{non-repudation}. Il server deve conservare al suo interno i dati medici degli utenti, in modo sicuro, in modo che siano accessibili soltanto da soggetti autorizzati. I dati potranno essere modificati, ad esempio aggiungendo nuovi esami fatti o medicinali prescritti. 
\end{description}

\section{Utilizzo del servizio}
Ogni utilizzatore del servizio dovrà essere equipaggiato con un dispositivo embedded, il quale ha lo scopo di identificare univocamente il possessore all'interno del sistema. In caso di emergenza, il soccorritore sarà munito di un altro dispositivo, compatibile con quello in possesso degli utenti, che consentirà di identificare in modo univoco il soggetto. Tramite questi dati sarà poi possibile per il soccorritore contattare un server remoto, ricevendo, se autorizzato, le informazioni mediche del soggetto identificato.

\section{Scenari}

\section{Requisiti non funzionali}
\begin{itemize}
	\item Sicurezza (dei dati e delle comunicazioni), resistente ad attacchi quali \emph{man-in-the-middle}, \emph{replay attack}, \emph{sniffing}\dots 
	\item \emph{Non-repudation} delle azioni fatte, identificazione univoca di tutti i soggetti (medico, utente). 
	\item Availability (del server e del dispositivo utente).
	\item Scalability del server
	\item Fault tolerance (server, batteria scarica dispositivo, mancanza di connessione internet).
\end{itemize}

\section{Domain model}

\section{Test plan}

\chapter{Analisi del problema}

\section{Studio di fattibilità}

\section{Abstraction gap}

\section{Problematiche}

\section{Analisi dei rischi}

\section{Architettura logica (?)}
Il sistema risulta essere composto dalle seguenti componenti:
\begin{itemize}
	\item Server
	\item Lettore
	\item Identificatore
\end{itemize}

\chapter{Piano di lavoro}

\chapter{Progettazione}

\chapter{Implementazione}

\chapter{Testing} 

\chapter{Deployment (?)}

\chapter{Manutenzione}

\chapter{Restrospettiva}
\section{Giudizio finale}
\section{Sviluppi futuri}

 
\end{document}
